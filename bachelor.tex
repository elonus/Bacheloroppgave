\documentclass{article}
\usepackage[utf8]{inputenc}

\usepackage[a4paper]{geometry}

\usepackage{amsmath}
\usepackage{amsthm} % theorems, definitions etc.

\usepackage{hyperref} % hyperlinks and pdf contents

\usepackage{graphicx}
\DeclareGraphicsExtensions{.png,.pdf}
\graphicspath{{images/}}

\usepackage[english]{babel}
\usepackage[square, numbers]{natbib}
\bibliographystyle{abbrvnat}
\setcitestyle{authoryear,open={((},close={))}}

%\usepackage[mmyyyy]{datetime}

\title{\textbf{Linear Regression for Survey Data} \\
{\Large Bachelor's Thesis}}
\author{Andreas Matre \\
Supervisor: Geir-Arne Fuglstad}
%\date{\today}
\date{May 2020}

\begin{document}

%\newtheorem{definition}{Definition}[section]
\newtheorem{definition}{Definition}
\newtheorem{theorem}{Theorem}
%\newtheorem{example}{Example}[section]
\newtheorem{example}{Example}


\maketitle

\begin{abstract}
  % 200 ord
  This Bachelor's thesis is submitted for the course MA2002 at NTNU which is 15
  credits over one semester.

  In this thesis we discuss how to do linear regression when the data is
  collected using a complex sampling design. We use a different paradigm from
  classical regression, where we assume an infinite population and that the
  value observed for each individual is random . Here, we instead acknowledge
  that the population is finite and assume the value of each individual is fixed
  and the randomness arises from which individuals to include in the sample. The
  major issues are accounting for different sampling designs to prevent bias and
  incorrect uncertainty estimates.

  We explain three different sampling techniques: the first sampling technique we look at is a Simple Random Sample. First, we
  choose the size of the sample. Then each possible subset of
  the population of that sample size has the same probability of being chosen as
  the sample. A Simple Random Sample has the advantage that all the sampled
  units are independent.
  The second sampling technique we look at is stratification. Here we split the
  population into a partition and sample independently from each subset. This
  allows us to get independent regression lines from each subset. The third
  sampling technique we look at is clustering. Here we again split
  the population into a partition, but instead of sampling from all subsets of
  the partition we instead sample only from some of the subsets, chosen by
  taking a sample of the subsets. Clustering is used
  to reduce costs when doing surveys. Often the clusters are geographical areas,
  which means that sampling only inside some subsets allow us to save 
  travel time. When performing large surveys these techniques are usually combined into what
  is called a complex survey, for example by first doing stratification on the
  whole population and then using clustering inside each subset.

  If the sampling units inside the strata are similar, stratification reduces
  the uncertainty compared to a SRS. With clustering, however, we usually get
  larger uncertainty, as units inside clusters are usually more similar than
  units across clusters. This causes the sample to carry less information than a
  non clustered sample. The non-linear
  nature of the regression coefficients means that estimating their variance
  becomes complicated. We therefore show an approximation technique called linearization.
\end{abstract}

\newpage

\tableofcontents

\newpage

\section{Introduction} \label{sec:intro}

%TODO: Jeg liker strukturen på og innholdet i introduksjonen og synes denne fungerer veldig bra. Det er noen formulering og valg av ord som du må jobbe litt med. Det er ikke greit å si “pulled from a distribution”. “Drawn from a distribution” er en mulighet. Men jeg vil at du jobber mer med teksten til du føler deg ferdig før jeg gir mer detaljerte tilbakemeldinger. \cite{st1101}

\begin{figure}
  \centering
  \includegraphics[scale = 0.5]{Residuals_for_Linear_Regression_Fit.png}
  \caption{Illustration of residuals. The red line is a regression line. The
    distances between the points and the line are illustrated by the vertical
    black lines.
    https://upload.wikimedia.org/wikipedia/commons/e/ed/Residuals\_for\_Linear\_Regression\_Fit.png
  }
  \label{fig:residuals}
  
\end{figure}

Linear regression is a useful tool to describe relationships between variables.
When we want to investigate possible associations, linear regression
is one of the most common tools we use. It can be used to show correlation between
variables, and in some cases, where experiments are designed very carefully, it can even suggest causation.
It can, for example, be used in the social sciences to try to show a relationship
between income and age of death. In simple linear regression, we want to determine the
line that best describes the relationship between the predictor and the
response. Since the relationship is usually not exactly linear, there is almost
always some noise around the line, which gives us residuals, the vertical
vectors that are between the points and the linear line. These are illustrated
in Figure \ref{fig:residuals}. If the relationship was
exactly linear we would just need to sample two individuals, which different
predictor value, to find the regression line. When doing linear
regression, the goal is to find the line that minimizes the sum of the residuals
squared. %TODO: Flytte siste setning til slutten av avsnittene om model based og
         %design based.

Since we are doing measurements on a real population, it is also finite.
If we knew the measurements for the whole population, there
would be no uncertainty. In that case we could just calculate whatever quantity we
wanted, including the regression intercept and slope. When we generalize from a sample to the whole population, however, we get
uncertainty because we can never know the values of the individuals not in the sample. There are different approaches of modeling this uncertainty, and
which method we should choose depends on what we know about the data. For example, which
methods were used in collecting the data?
What is the probability of each
individual being included in the sample? Are the individuals sampled independently? 

One approach to model the data is called the model based approach and is based
on the fact that we usually sample from large populations. We assume that the
relationship between the predictor and the response is split in two: a linear deterministic
part and a stochastic part representing the error, which can be described by a continuous distribution. We assume that the population
is large enough that the probability of sampling the same unit again if we take
a new sample is negligible. This allows us to assume that when we sample an unit, 
the response is the value of the line, given the predictor value, pluss the stochastic
part, called the residual, which is drawn from the distribution. We can use the
structure from assuming that the residuals are drawn from a distribution to
model the uncertainty of the coefficients and predictions. The only thing we need
to know about the data collection method is the fact that the individuals were
sampled independently. In the model based approach, we assume there exists a
line representing the deterministic part of the relationshiop between the
predictor and response. Our goal is to use a sample of the population to get a
estimate of the true line. The assumption that the
probability of sampling the same unit again is negligible if we resample is often not
realistic, however, nor the fact that the residuals follow a
distribution. This motivates another approach to model the data. 


% One approach to model the uncertainty is called the model based approach, which
% is based on the fact that we usually sample from large populations. If the
% population is large enough, taking a sample is almost the same as drawing
% values from a distribution around the regression line. This could be any stochastic distribution, but the most
% common one is the normal distribution. The fact that we assume that the
% sample is drawn from a distribution means that we assume that the value of an
% individual can change if we sample them again. The assumption that the residuals
% are drawn from a distribution implies a structure to the residuals, which we
% can use to model the uncertainties in the residual line. The only thing we need
% to know about the data collection method is the fact that the individuals were
% sampled independently. The fact that the observed values can change is not that
% realistic, however, nor the fact that the population approximates a
% distribution around the regression line. These considerations
% motivate another approach.

%TODO: Skriv om litt når jeg får tilbakemelding på forrige avsnitt om model based
A perhaps more realistic way to model the data is to assume that every time we measure an
individual, we observe the same response, in which case we can not pretend to be drawing
values from a distribution. To model the uncertainty in this approach, we look at
the sampling probabilities, i.e, the probability of each individual being
included in the sample. Or, said another way, the random part of the model
is who is sampled as opposed to the values of the sampled units. Knowing the sampling probabilities is enough to
estimate the regression intercept and slope, but to estimate the uncertainties,
we also need information regarding hot the sample was collected.
Using this type of information and the fact the the random part is now who is in
the sample is called a design based approach. The design based approach requires
more information about the sampling scheme than the model based approach. The
estimated variances with a design based approach are often larger than when
using a model based approach. This is because the
assumption that the observations are drawn from a specific distribution gives a
lot of extra structure. In the design based approach our goal in regression is
to find the line minimizing the residual sum of squares for the whole population.

The population we are sampling from could be heterogenuous, i.e, different
parts of the population have very different response values. Say, for example, we want
to find the mean income of residents of Oslo. When sampling, we could risk
getting only people living in the west part of Oslo, which is the most wealthy
area of the city. This would result in the estimate being much higher than the true
mean income of the residents of Oslo. To fix this, we could split
the city population into subsets; one for each city district. We can then sample
from each of these districts independently. Doing that, we are guaranteed to get
a sample including people from different parts of the city, and therefore
more likely to get correct estimates. This can dramatically decrease the
uncertainty of the estimates when the subsets are chosen smartly. Doing this also allows us to make
separate estimates of the quantity of interest for different parts of the
population, i.e, we could make a separate estimate of the mean income for each city
district in addition to the estimate for the city as a whole. We may also be
interested in making separate analyses on different parts of the population. We
might, for example, be interested in separate means of income for different
ethnicities. This can also be done by splitting the population into subsets to
make sure that we have enough data about each group we are interested in. This method is
called stratification.


Another potential problem is that researchers are often on a limited budget, and
it can be expensive to sample randomly from the whole population. Say
for example that 
each sampled individual requires an interviewer to show up personally. If the sample is spread randomly in the country, this can get
expensive as the interviewers would need to use much time to travel. One often solves
this by splitting the country into geographic parts, for example municipalities, and
then randomly choose some municipalities to sample people from. One then pick
independent samples from the chosen municipalities. This has the advantage that it
saves travel time, and, therefore, makes it cheaper to conduct the
survey. The problem with this approach, however, is that the selection of units become
dependent since we sample from only some municipalities. This results in larger
variances. This can be remedied by the fact that we can sample more
individuals inside just some municipalities for the same
cost as having a smaller sample from the whole
population. This is called clustering.

We often combine stratification and clustering, which results in complicated
sampling schemes where we, for example, first cluster on municipalities, then within
each municipality we may, for example, split the population by age and sample from all
the age groups. Because this gets complicated, it is often difficult to
find explicit formulas for the variance, and we usually have to estimate the variance
instead. 
The fact that we have a finite population, also influences the variance, since we
get something called the finite population correction. This is a factor \(1 -
\frac{n}{N}\), where \(n\) is the size of the sample and \(N\) is the size of
the population. This factor takes into account the fact that as we get a larger
and larger sample we can learn everything about the population, and therefore
the variance goes to zero.

This thesis shows how to fit
a linear regression model when data is collected through a complex survey design, i.e,
a survey including unequal sampling probabilities, stratification and
clustering. To do analysis in these cases weights are used. Each observation gets a weight
value which can be interpreted as the number of individuals in the population
that observation represents. So a unit with a small chance of being
included in the sample would have a larger weight than a unit with a high chance of
being included.
In classical regression each individual in the population has the same chance of
being included so each observation therefore has the same weight.

We start
with an example illustrating what can go wrong if the design used in collecting
the data is not taken into account.


\begin{example} \label{ex:anthuneq}

We use a dataset from a study of the relationship between the length of a persons left middle finger and their height. 
The researcher oversampled people having short fingers and undersampled people
having long fingers.
The dataset contains 200 samples, each containing the length of the persons left middle finger (cm), their height (cm) and the probability that they would be chosen for the sample.

To illustrate the difference, the top left plot in Figure \ref{fig:ex1} shows a random sample where every
person had an equal chance of being included. While the top right plot in Figure
\ref{fig:ex1} shows the sample
where short people had a higher chance of being included than tall people. The
observations in the right panel is much more concentrated in the bottom.
This means that fitting a linear regression model to the unequal probabilities
sample will result in the slope tending to be smaller than it should be.

\begin{figure}
  \centering
  
  \includegraphics[scale = 0.5]{ex1}

  \caption{The top row shows the two samples from the finger length versus
    height dataset. The top left plot shows a sample where everyone has the
    same chance of being included in the sample.
    The top right plot shows a sample where people with short fingers are
    oversampled. The bottom row shows estimated regression lines based on the
    samples. Each plot shows estimated
    regression lines for the sample in the plot above. The red lines are
    estimated regression lines using classical regression while the blue lines
    are estimated
    regression lines taking the probabilities of being included in the sample
    into account. The shaded areas represent the 95\% prediction intervals.
    Observe that in the bottom left plot the two approaches agree while in the
    bottom right plot, they do not agree.}

  \label{fig:ex1}
\end{figure}

%TODO: Prøv å forkorte avsnitt mer
The bottom row of Figure \ref{fig:ex1} illustrates the difference between
classical regression, assuming a normal distribution for the residuals,
and regression taking the sampling probabilities into account. The bottom left plot shows regressions based on a
sample where every person has the same probability of being included. We can see
only one regression line in this plot, this is because both regression lines are
exactly equal in this case. In the bottom right plot of Figure \ref{fig:ex1}, we
see there are two different lines: the red one is the regression line from
classical regression while the blue line is from regression taking sampling
probabilities into account. The shaded areas represent 95\%
prediction intervals. We can see that the red line seems to fit the sample much
better than the blue one which seems too steep. This is because the
blue line takes the oversampling of people with short fingers into account. When we look at the prediction
intervals we can see that the blue prediction interval seems to be ``correct''
by the metric that about 95\% of the dots are inside it, but the red one is far
from having that many dots.

These differences show that it is important to take the design of the survey
into account when analyzing the data, because classical linear regression can
give misleading estimates.
\end{example}

The rest of the thesis focus's on a dataset on the performance of students in schools in
California. The dataset has data on all 6194 schools that have more than 100
students in California. The data collected
includes: API scores in 1999 and 2000, which level of school it is
(elementary, middle, high), name of school, location of school, percentage of
students tested at the school, API targets, economic factors for students at the
school, class sizes, information of education of parents and qualification of teachers.
API scores is a metric quantifying the academic performance of the students at the school.
We will use this data as the population we will sample from, and we will take
different kinds of samples to illustrate the concepts introduces in this thesis,
\cite[Section 1.2]{complexR}.

In this thesis we will use the ``survey'' package in R to make computations \cite{surveyR}.

Section \ref{sec:modLinReg} will give a overview of model based simple
linear regression, which is assumed known and only a brief repetition will be given.
Section \ref{sec:RegFinPop} will explain how to do linear regression in the context of finite
populations using survey statistics. Section \ref{sec:SRS} will show how to do linear
regression when we have a Simple Random Sample. Section \ref{sec:strat} will go through
linear regression when we have a sample using stratification. Section \ref{sec:clustering} will
explain linear regression when we have a sample using clustering and Section \ref{sec:complexSurveys}
will show how to do linear regression when you have surveys combining
stratification and clustering into a complex survey. Section \ref{sec:VarEst} is
about estimating the variance of non-linear expressions while Section
\ref{sec:discussion} is the discussion.

\section{Classical simple linear regression} \label{sec:modLinReg}

In classical simple linear regression, each observation \(i\) consists of a
response variable, \(y_i\), and a predictor, \(x_i\) for \(i = 1, ..., n\), where \(n\) is the number of observations. The
relationship between the response and the covariates is assumed to be
\begin{equation*}
Y_i = \beta_0 + \beta_1 x_i + \epsilon_i,\ i = 1, \dots, n,
\end{equation*}
where \(\epsilon_1, \epsilon_2, \dots, \epsilon_n\) are stochastic variables
describing the error and the intercept, \(\beta_0\), and the slope ,\(\beta_1\),
are constants that describe the deterministic part of this relationship.
We do not know the values of \(\beta_0\) and \(\beta_1\), so we have to estimate
them from observed data points.

We call the data points \((x_1, y_1), (x_2, y_2)
... (x_n, y_n)\). To estimate the deterministic part of the relationship we want
to find estimators \(\hat{\beta}_0\) and \(\hat{\beta}_1\) that minimize the
Residual Sum of Squared (RSS). The RSS is defined by
\begin{align*}
  \mathrm{RSS} &= \sum_{i = 1}^n \left( y_i - \hat{y}_i \right)^2 
  = \sum_{i = 1}^n \left( y_i - \hat{\beta}_0 - \hat{\beta}_1 x_i \right)^2.
\end{align*}
RSS can be geometrically thought of as the sum of the squared distances of
the data points in the sample from the regression line, see Figure
\ref{fig:residuals}. We want the line to follow the linear trend in the sample
as closely as possible, and minimizing the RSS is one way to find a close line.

The estimators \(\hat{\beta}_0\) and \(\hat{\beta}_1\) that minimizes the RSS are
%$$
%\hat{\beta}_1 = \frac{\sum_{i = 1}^n\left( x_i - \bar{x} \right) \left( y_i -
    %\bar{y} \right)} {\sum_{i = 1}^n \left( x_i - \bar{x} \right)}
%$$
%and
%$$
%\hat{\beta}_0 = \bar{y} - \hat{\beta}_1 \bar{x}
%$$

\begin{equation} \label{eq:beta1}
 \hat{\beta}_1 = \frac{\sum_{i = 1}^n\left( x_i - \bar{x} \right) Y_i}{\sum_{i = 1}^n\left( x_i - \bar{x} \right)^2} ,
\end{equation}

\begin{equation} \label{eq:beta0}
 \hat{\beta}_0 = \bar{Y} - \hat{\beta_1}\bar{x},
\end{equation}
where \(\bar{x} = \frac{1}{n} \sum_{i = 1}^n x_i\) and \(\bar{y} = \frac{1}{n}
\sum_{i = 1}^n y_i\). \cite[Chapter 11]{st1101}

Assume the following conditions:

\begin{enumerate}
\item \(\mathrm{E} \left( \epsilon_i \right) = 0\ \forall \ i = 1, 2, ..., n\)
\item \(\mathrm{Var} \left( \epsilon_i \right) = \sigma^2\ \forall \ i = 1, 2, ..., n\)
\item All the \(\epsilon\) are independent of any predictor or observation number.
\item All \(\epsilon_1\), \(\epsilon_2\), ..., \(\epsilon_n\) are independent of each other,
\end{enumerate}

%TODO: I Section 2 I teksten som følger etter de fire ligningene øverst på side 6, må du være mer korrekt. Du skriver “… as likely to be over the line as under it”. Dette er ikke riktig med mindre fordelingen til residualene er symmetrisk. Du burde skrive “… on average lie on the line” eller noe sånt. På samme måte må du være mer presis med utsagnet om item 2.


Here item 1 means that data points should on average lie on the line. \\
Item 2 means that the data points should have roughly the same vertical
Euclidean distance from the regression line. \\
Item 3 and 4 means that there should be no pattern in whether the data points is over
or under the regression line and on the vertical Euclidean distance from the
regression line.

Under these conditions we get some useful properties, including that \(\hat{\beta}_0\) and \(\hat{\beta}_1\) are unbiased estimators of
\(\beta_0\) and \(\beta_1\). In addition we have
unbiased estimates for the
variance of the estimators \(\hat{\beta}_0\) and \(\hat{\beta}_1\)
\begin{equation*}
 \widehat{\mathrm{Var}} \left( \hat{\beta}_0 \right) = \frac{1}{n - 2} \sum_{i = 1}^n\left( y_i - \hat{\beta}_0 -
 \hat{\beta}_1 x_i \right)^2 \frac{\sum_{i = 1}^n x_i^2}{n
   \sum_{i = 1}^n \left( x_i - \bar{x} \right)^2}
\end{equation*}
 

\begin{equation*}
 \widehat{\mathrm{Var}} \left( \hat{\beta}_1 \right) = \frac{1}{n - 2} \sum_{i = 1}^n\left( y_i - \hat{\beta}_0 -
 \hat{\beta}_1 x_i \right)^2\frac{1}{
   \sum_{i = 1}^n \left( x_i - \bar{x} \right)^2},
\end{equation*}

where \(\frac{1}{n - 2} \sum_{i = 1}^n\left( y_i - \hat{\beta}_0 -
 \hat{\beta}_1 x_i \right)^2\) is an
unbiased estimator for the unknown \(\sigma^2\), \cite[Chapter 11]{st1101}.

This is based on the fact that we assume the response is random, i.e, that
it is possible to get a different response if we measure again. This means we
can think of the residuals as coming from some stochastic distribution. We will
now consider what happens if we instead assume that the responses are fixed and the
randomness comes from how we select individuals instead.
This approach is called survey statistics


\section{Regression in the context of finite populations} \label{sec:RegFinPop}

%TODO: Riktig subsection tittel?
\subsection{Introduction}

Now consider the case where the population is finite, i.e, the population
consists of \((x_1, y_1),
(x_2, y_2), \dots , (x_N, y_N)\), where the \(x_i\)'s are the covariates we use to
predict \(y_i\) and \(N\) is the size of the population.
The goal of linear regression in this case is that we want to find the
line, \(y = B_0 + B_1x\), that best describes the relationship between \(x_i\)
and \(y_i\) in this population. We define the best line as the one
that minimizes \(\mathrm{RSS} = \sum_{i = 1}^N (y_i - B_0 - B_1 x_i)^2\).
Minimizing the RSS means that we minimize the the squared Euclidian distance of each point
in the population from the line.

This differs from model based linear regression where we want to estimate the
deterministic part of the relationship between \(x_i\) and \(y_i\), but we can
never get an exact answer as the estimates will differ when they are based on
different samples, no matter how large. Here, however, it is possible to find \(B_0\) and
\(B_1\), because we can simply sample the whole population, even if that typically
is not possible in practice.

If we, however, knew the whole population, we could just compute \(B_0\) and \(B_1\) using the same formulas
as in Section \ref{sec:modLinReg}. For our purposes we will rewrite Equations
\ref{eq:beta1} and \ref{eq:beta0} a bit,
so that they are expressed by totals of the population. They therefore become
\begin{equation} \label{eq:B0}
 B_0 = \frac{1}{N} \left( t_y - \frac{t_{xy} t_x - \frac{1}{N} t_y t_x^2}
   {t_{x^2} - \frac{1}{N} t_x^2},
  \right)
\end{equation}

\begin{equation} \label{eq:B1}
 B_1 = \frac{t_{xy} - \frac{1}{N} t_y t_x}
   {t_{x^2} - \frac{1}{N} t_x^2},
\end{equation}
where \(t_x = \sum_{i = 1}^N x_i\), \(t_y = \sum_{i = 1}^N y_i, t_{x^2} =
\sum_{i = 1}^N x_i^2\) and \(t_{xy} =
\sum_{i = 1}^N x_i y_i\), \cite[Chapter 11]{sampReg}.

The case where we know the whole population is not realistic. We are interested in the case where we have to sample from the
population to estimate \(B_0\) and \(B_1\). To do that we have to introduce some terms.


\begin{definition} \label{def:sampUnitPop}
 A \textbf{sampling unit} is one "element" we can sample.
 The \textbf{sampling population}, or universe, \(U = \{1, 2, 3, ..., N\}\), is a
 finite set containing all the sampling units we can sample.
\end{definition}

In the case of our API dataset, our sampling units are schools, while
the sampling population is all the schools in California having more than 100 students.

\begin{definition} \label{def:sampFrame}
 A \textbf{sampling frame} is a list of sampling units that one uses to draw a sample.
\end{definition}

The sampling frame would be all schools in California that the researchers know
about and that the researchers think have more the 100 students.
Ideally the sampling frame and the sampling population would be the same, but that is not always to case.
When taking different types of samples from the API dataset, the sampling population and the sampling frame are equal, since
we have a table of all the data and we will just choose rows from that table for
our samples. But there are cases where this is not the case. An example of this
could be if we were doing a political survey to try to predict who will win the
next election.
In that case our sampling population, who we are interested in information
about, would be everyone who are going to vote in the next election. It is
impossible to get a list of them though, so we instead have to use 
some other part of the population we have information about as our sampling frame. We might have a
list of all who voted last election, which might be a good approximation of
those who are going to vote this election, but then we would miss out on all the
new eligible voters and people who might have decided to vote this election but
didn't do it the last one.
Choosing the correct sampling frame to match your target sampling population is
difficult and if they do not match it may influence the results.

We will represent an unique integer index to each sampling unit and list them in an arbitrary order. This simplifies notation.


\begin{definition} \label{def:sample}
A \textbf{sample}, \(S \subseteq U\), is a subset of the sampling frame. This is the data we will analyse to learn about the sampling population.
A \textbf{probability sample} is a sample where the sampling units included are chosen randomly.
The \textbf{sampling probability} of a sampling unit is the probability that a
specific sampling unit will be included in the sample.
\end{definition}


If we then let \(\hat{t}_x, \hat{t}_y, \hat{t}_{x^2}, \hat{t}_{xy}, \hat{N}\) be estimators
for \(t_x, t_y, t_{x^2},
t_{xy}, N,\) respectively, we get the estimators for \(B_0\) and \(B_1\) by
replacing quantities by estimated quantities in Equations \ref{eq:B0} and
\ref{eq:B1} respectively
\begin{equation*}
 \hat{B}_1 = \frac{\hat{t}_{xy} - \frac{1}{\widehat{N}} \hat{t}_y \hat{t}_x}
   {\hat{t}_{x^2} - \frac{1}{\widehat{N}} \hat{t}_x^2}
\end{equation*}

\begin{equation*}
 \hat{B}_0 = \frac{1}{\widehat{N}} \left( \hat{t}_y - \frac{\hat{t}_{xy} \hat{t}_x - \frac{1}{\widehat{N}} \hat{t}_y \hat{t}_x^2}
   {\hat{t}_{x^2} - \frac{1}{\widehat{N}} \hat{t}_x^2}
 \right)
 = \frac{\hat{t}_y}{\hat{N}} - \hat{B}_1\frac{\hat{t}_x}{\hat{N}}
\end{equation*}

\textbf{Fjerne dette avsnittet om varians og bare refere til seksjon 4?}\\
Since \(\hat{B}_0\) and \(\hat{B}_1\) are non linear expressions of dependent
statistics, deriving exact expressions for the variances is complicated. We, therefore,
often have to settle with having estimates of the variances instead.
There are several ways to do so, but a common one,
and the one we use in this thesis, is linearization. Linearization takes a
non-linear expression of stochastic variables we want to do inference about and uses
the first two terms of the Taylor expansion to make it linear.
One can shot that
\begin{align*}
 \mathrm{Var}(\hat{B_1})
 &\approx \frac{\widehat{\mathrm{Var}}\left( \sum_{i \in S} w_i q_i \right)}
   {\left( \sum_{i \in S} w_i x_i^2 - \frac{\left( \sum_{i \in S} w_i x_i \right)^2}{\sum_{i \in S} w_i} \right)^2}
\end{align*}
where \(q_i = (y_i - \hat{B}_0 - \hat{B}_1 x_i)(x_i - \hat{\bar{x}})\).
\(\widehat{\mathrm{Var}}\left( \sum_{i \in S} w_i q_i \right)\) is easier to
work with as \(\sum_{i \in S} w_i q_i\) estimates a total. See \cite[Chapter
11.2.2]{sampReg} for details on derivation. We describe linearization in more
detail in Section \ref{sec:VarEst}.

%TODO: Si noe sånt som: Vi må nå vite hvordan vi kan estimere totalene for
%forskjellige samples.

\subsection{Simple random sample} \label{sec:SRS}

The simplest probability sample is the Simple Random Sample (SRS). A
sample of size \(n \leq N\) is an SRS if every subset \(S \subseteq U\) has the same
probability of being chosen.
If, for example, \(U = \{1, 2, 3, 4\}\) and we want a sample \(S\) of
size \(3\), then there are \(\binom{4}{3} = 4\)  possible samples:
%\begin{equation*}
%S_1 = \{1, 2, 3\}, 
%S_2 = \{1, 2, 4\}, 
%S_3 = \{1, 3, 4\}, 
%S_4 = \{2, 3, 4\} .
%\end{equation*}
\( S_1 = \{1, 2, 3\},\ \)
\( S_2 = \{1, 2, 4\},\ \)
\( S_3 = \{1, 3, 4\}\ \) and
\( S_4 = \{2, 3, 4\} \).

For this to be a SRS each of these subsets need to have the same probability of
being chosen, i.e, \(P(S_1) = P(S_2) = P(S_3) = P(S_4) = 0.25\). A consequence of
having a SRS is that all the sampling probabilities are equal, \(P(1 \in S) =
P(2 \in S) = P(3 \in S) = P(4 \in S) = 0.75\). But having equal sampling
probabilities is not sufficient for the sample to be an SRS.
Look, for example, at this case:
Assume we want a sample of size \(2\) from a population of size \(4\), and that
\(P(\{1, 3\}) = 0.5\) and \(P(\{2, 4\}) = 0.5\) while the probabilities of all the
other possible samples are \(0\). Then \(P(1 \in S) = P(2 \in S) = P(3 \in S) = P(4 \in S) = 0.5\)
but this is not a SRS since all possible subsets of size \(2\) do not have equal
probability of being chosen. This is actually an example of a cluster sample
which is discussed in Section \ref{sec:clustering}.

We need estimates of several different totals of the population to estimate the
regression coefficients, see Equations \ref{eq:B0} and \ref{eq:B1}. We need the total of, among others, the \(y_i\)'s, the
\(x_i\)'s and the \(x_i y_i\)'s, to calculate the estimates.
We will do inference on the total of the \(y_i\)'s here. The other totals
are equivalent and not shown.

Let
\(
 t_y = \sum_{i = 1}^{N} y_i
\)
be the value we want to estimate.
The natural estimator of this total, if we sample \(n\) elements, would be
\(
\hat{t}_y = \frac{N}{n}\sum_{i \in S} y_i
\)
where we take the average of the values in our sample and then scale it up to
the whole population.
It can be shown using indicator variables and sampling probabilities that
\(\hat{t}_y\) is an unbiased estimator for \(t_y\).

It can also be shown that the variance of the estimator is of the form, see
\cite[Chapter 2]{sampReg} \begin{equation*}
\mathrm{Var} \left( \hat{t}_y \right) = \frac{N^2}{n \left( N - 1 \right)} \left( 1 - \frac{n}{N} \right) \sum_{i = 1}^N (y_i - \bar{y})^2 
\end{equation*}

where
\(
\frac{1}{N - 1} \sum_{i = 1}^N (y_i - \bar{y})^2
\)
is the variance of the whole population.
We do not, however, know the population variance, as that would require us to know the \(y\) values
for the whole population \(U\). Instead we estimate the population variance by the unbiased estimator\(
 \frac{1}{n - 1} \sum_{i \in S} \left( y_i - \hat{\bar{y}} \right)^2
\)
, where \(\hat{\bar{y}} = \frac{1}{n} \sum_{i \in S} y_i \), which gives us the estimate of \(\mathrm{Var}(\hat{t}_y)\)\begin{align*}
 \widehat{\mathrm{Var}(\hat{t}_y)}
 &=\frac{N^2}{\left( n - 1 \right)n} \left( 1 - \frac{n}{N} \right) \sum_{i \in S} \left( y_i - \hat{\bar{y}} \right)^2
\end{align*}
Since \(\frac{1}{n - 1} \sum_{i \in S} \left( y_i - \hat{\bar{y}} \right)^2\) is
an unbiased estimator of \(\frac{1}{N - 1} \sum_{i = 1}^N (y_i - \bar{y})^2\), we have that \(\widehat{\mathrm{Var}(\hat{t}_y)}\) is an unbiased estimator of the variance of \(\hat{t}_y\).

We see that all these estimators for totals and means are the same as in the
model based case. Therefore, the estimates for the coefficients in the regression
model are also the same. The variance of the estimators for the coefficients are
different however.
The factor \(\left( 1 - \frac{n}{N} \right)\) in the variances is what differs in
the variance estimate compared to the model based one. It is called the
\textbf{finite population correction (fpc)} and comes from the fact that we are
sampling without replacement from a finite population.
For an intuitive explanation of the fpc, consider we take a sample of size
\(10\). If the population size is just \(15\) we would expect to have a lot more
information about the whole population than if the population size was large.
The fpc also makes sure that the variance is \(0\) if we sample the whole
population. Note, however, that if we do prediction using our regression line
we will have a deterministic model. This is because the data points lay around
the line, so there will always be uncertainty when predicting.

Using linearization we get this estimate for the variance of \(\hat{B}_1\) when
the sample is from an SRS, see \cite[Chapter 11.2]{sampReg}  \begin{equation*}
\widehat{\mathrm{Var}}(\hat{B}_1) = \left( 1 - \frac{n}{N} \right) \frac{n}{n - 1} \frac{\sum_{i \in S} \left( x_i - \bar{x} \right)^2 \left( y_i - \hat{B}_0 - \hat{B}_1 x_i \right)^2}
{\left( \sum_{i \in S} \left( x_i - \bar{x} \right)^2 \right)^2}
\end{equation*}

which we can compare to the estimate of the variance for \(\hat{\beta}_1\) in
the model based case
\begin{equation*}
 \widehat{\mathrm{Var}} \left( \hat{\beta}_1 \right) = \frac{\sum_{i = 1}^n\left( y_i - \hat{\beta}_0 -
 \hat{\beta}_1 x_i \right)^2}{
   \left( n - 2 \right)\sum_{i = 1}^n \left( x_i - \bar{x} \right)^2},
\end{equation*}
where \(\bar{x} = \frac{1}{n} \sum_{i \in S} x_i\).
%TODO: Write some words to compare


\subsection{Stratification} \label{sec:strat}

In stratification we split the sampling frame into a partition, i.e, \(H\) non-
overlapping subsets that together comprise the whole sampling frame. These
subsets are called \textbf{strata}. We let each stratum have \(N_i,\ i = 1, \dots
, H,\) elements. Thus \(N_1 + N_2 + \dots + N_H = N\). When sampling we
independently draw samples from each stratum, \(S_1, S_2, \dots, S_H\), with \(n_i,\ i = 1, \dots
, H,\) elements. When
estimating a total we can first estimate the total of each stratum and
then add these estimated totals to get an estimate of the population total.
If we let \(t_{y,h}\) be the total in stratum \(h\), for \(h = 1, 2, \dots, H\), we get \(\hat{t}_y =
\sum_{h = 1}^H \hat{t}_{y, h} \), where we can use different sampling schemes to
estimate each \(t_{y, h}\).
If we let each sample of the subsets be a simple SRS, we get \( \hat{t}_y =
\sum_{h = 1}^H\sum_{i \in S_h}\frac{N_h}{n_h}y_i\).
Since the estimate for each
stratum total is unbiased (see Section \ref{sec:SRS}), the estimate of the population
total is unbiased.

Since we often sample differently in the different strata,
the individuals sampled from the different stratum usually have
different sampling probabilities. This means that different sampling units
should be weighted differently when making estimates, as illustrated in Example \ref{ex:uneqProbs}.

\begin{example} \label{ex:uneqProbs}
  Suppose a population is divided into two strata, each with a subpopulation of
  \(1000\) individuals. Let one subpopulation be values with mean
  \(0\) and variance \(1\) and one subpopulation be values with mean \(10\) and variance \(1\), called
  \(A\) and \(B\).
  Our goal is to estimate the sum of the values.

  The true sum of the population values is \(10058.59\). If we sample the same
  proportion of values from \(A\) and \(B\) we can estimate the sum the same way
  as in a SRS. Let \(S_{A, 100}\) and
  \(S_{B, 100}\) be SRS's of size \(100\) from \(A\) and \(B\) respectively.
  Then an unbiased estimate of the total is \(\sum_{i \in S_{A, 100} \cup S_{B,
      100} } \frac{2000}{200} y_i = 10104.81\). The bias is only \(46.22\) which is only \(0.46\%\) of the
  value.

  If the proportion of units sampled from \(A\) is different from the
  proportion of units sampled from \(B\), however, we
  can't use this simple estimate. Let \(S_{A, 50}\) be a SRS of size \(50\)
  from \(A\). If we use the same SRS estimate again we get \(\sum_{i \in S_{A, 50} \cup S_{B,
      100}} \frac{2000}{150} y_i = 13560.02\). The bias is here \(3501.425\) which is \(34.8\%\) of the
  value. This is a much larger error. This is because we have more samples from
  \(B\) than \(A\), and \(B\) has a higher mean than \(A\). Since each sampled
  value is counted the same this causes the estimate to become too large. To fix
  this we need to let the sampled values from \(A\) count, or weigh, more than
  the ones from \(B\). This is because the values in \(S_{A, 50}\) have to
  represent the same size population as the values in \(S_{B, 100}\), but there
  are only half as many values in \(S_{A, 50}\) as in \(S_{B, 100}\). Therefore,
  each value in \(S_{A, 50}\) has to count two times as much as the values in \(S_{B, 100}\).

  Doing this the estimate becomes \(\sum_{i
    \in S_{A, 50}} \frac{1000}{50} y_i + \sum_{i \in S_{B, 100}}
  \frac{1000}{100} = 10174.61\) which has a bias of \(116\) which is \(1.15\%\),
  a huge improvement over the \(34.8\%\) bias where we did not take the
  different sampling probabilities into account.
\end{example}

This illustrates how important it is to take the sampling probabilities into
account. To do this we usually give each sampled unit a weight value,

\begin{definition}
 The \textbf{weight} of a sampling unit is the inverse of the sampling
 probability of the sampling unit. 
 We denote the weight of unit \(i\) as \(w_i\).
\end{definition}

An intuitive way to interpret weights is that the weight of an observation is
how many sampling units in the population they represent. If we sample few units
from a large strata, each of these sampled units represents many more individuals than
in a strata where we sample almost the whole subpopulation. The name weight
makes sense, as an observation representing many unobserved units should
``weigh'' more when estimating values than observations representing few
unobserved units.

Specifically, in Example \ref{ex:uneqProbs}, the \(50\) sampled units from \(S_{A,
  50}\) represent a population of \(1000\), so each sampled unit represents
\(20\) units including itself. This is opposed to the \(100\) sampled units in
\(S_{B, 100}\) which also represent a population of \(1000\). Here each sampled
unit only represents \(10\) units including itself.


Using weights, we can rewrite the estimate of \(t_y\) in a more general form,
which is valid for any sampling scheme, \(\hat{t}_y = \sum_{i \in S}w_i y_i\),
where the full sample is \(S = S_1 \cup S_2 \cup \dots \cup S_H\).
This is a convenient way to calculate estimates of more complicated surveys, as
one only needs to calculate the weights once, and then one can use them to
estimate many different quantities.
This also works for an SRS as it can be shown that the sampling probability of a sampling unit in a SRS is
\(\frac{n}{N}\). This means that the weight of each sampling unit is
\(\frac{N}{n}\). Therefore the estimator using weights is the same as the one
introduced in Section \ref{sec:SRS}, \(\hat{t}_y = \sum_{i \in S}\frac{N}{n} y_i\).

Since the samples from the different strata are independent, the variance of the estimator is also
easy to calculate, \(\mathrm{Var}(\hat{t}) = \mathrm{Var}\left(\sum_{h =
   1}^H\hat{t}_{y, h}\right) = \sum_{h =
   1}^H\mathrm{Var}\left(\hat{t}_{y, h}\right)\). This
means that to minimize the variance of \(\hat{t}\) we should choose the strata
such that the internal variance in each stratum is as small as possible.

Stratification is used for several reasons: making it possible to analyze
subpopulations individually, making sure subgroups are included in the sample,
and reducing uncertainty.

If we have a population with several different interesting subgroups, it can be
useful to let each of these subgroups be their own strata. This will allow us to
create a seperate regression line for each stratum, which will allow us to
analyze them by themselves. We can then compare the slopes to see if there is a difference in the relationship between the
response and predictor for the different strata. We could of course make
regression lines for different subgroups after doing a sample without stratification,
but then we would have no guarantee that each of the subgroups would have enough
samples to be able to make a useful regression line. Using stratification, we can
choose how many samples we want from each subgroup.

\begin{example}
  Suppose that we want to investigate the relationship between the API score of
  a school and the average level of education of the student's parents. We might
  be interested in knowing if the effect the parents education level has on
  their childs performance changes as they get older. It would therefore make
  sense to stratify on which level the school is, elementary school, middle
  school or high school. We can then see if the estimated slope coefficient is
  different in each of these strata.
  The sampling frame has \(4421\) elementary schools, \(755\) middle schools and
  \(1018\) high schools and we choose to sample \(50\) schools from each strata.

  Table \ref{tab:exStrat} shows the estimated slope with 95\% confidence intervals along with
  true slopes for each school level. We observe that all the true values are
  inside the confidence intervals. We can see that
  parent education level seems to have a higher effect in Elementary schools
  than higher school levels. Regression on all the strata together gives the
  slope \(158\), which is somewhere in between all the individual slopes. Not
  doing stratification would make us loose the information for each
  school type.

  \begin{table}
    \centering
  \begin{tabular}{c|ccc}
    & Elementary schools & Middle schools & High schools \\
    \hline \\
    Estimates (95\% confidence interval) & 169.1 (136.9, 201.3) & 145 (121, 169) & 143.8 (119.7, 167.9) \\
    True value & 144.5 & 157.7 & 152.2
  \end{tabular}
  \label{tab:exStrat}
  \caption{Table of slope coefficients for the different school levels. The
    first row has estimates from the sample, with 95\% confidnce intervals in
    parenthesis. The second row has the values, as we know the whole population.}
  \end{table}
\end{example}


\subsection{Clustering} \label{sec:clustering}


In clustering we split the sampling frame into a partition as in stratification.
Here, however, we do a probability sample to choose which of the subsets we will
collect data from. Then we have to do a probability sample
inside of each of these chosen subsets, \(S_1, S_2, \dots, S_n\).

\begin{definition}
 \textbf{Primary sampling unit (psu)} are subsets of the sampling frame, that
 are sampled first in a sampling scheme. These are also often called \textbf{clusters}. 
\end{definition}

There are two ``types'' of cluster sampling; one-stage cluster sampling and
multi-stage cluster sampling. In one-stage cluster sampling we sample all the
elements in the subsets chosen of \(S\), so each element has probability \(1\)
of being included. In multi-stage cluster sampling, however,
we make a sample of the \textbf{secondary sampling units (ssus)}, which are
individuals inside the clusters. Here not all ssus, or
individuals, inside the clusters are included. Multi-stage cluster sampling is often what is used in
practice, but since the ideas are similar to one-stage cluster sampling and
the formulas get much more complicated in multi-stage cluster sampling,
we will restrict outselves to one-stage cluster sampling in this thesis.

Let \(N\) be the number of clusters (psus) in the population and let \(M_i,\ i =
1, 2, \dots, N\) be the number of individuals (ssus) in each cluster. Let
\(t_{y, i}\) be the total for cluster \(i\). Since we sample all the ssus inside
the clusters, we know the totals for the
clusters in the sample. We can therefore consider each cluster a sampling unit as we
have done earlier in the thesis. Instead of letting the \(y_i, i \in S\) be the
sampled values, we let \(t_{y, i}, i \in S\) be the sampled values. In one-stage cluster sampling, an SRS is
usually taken to choose which clusters to include in the sample. This gives a
similar estimator for the total as for a SRS, except that the sampling units here
are the totals for each cluster instead of individuals as in a normal SRS,
\(\hat{t}_y = \sum_{i \in S} \frac{N}{n} t_{y, i} = \sum_{i \in S} w_i t_{y, i} \). 
This also gives the same estimate for the variance as in an SRS,
\(\widehat{\mathrm{Var}}\left(t_{y}\right) = \frac{N^2}{n} \left( 1 - \frac{n}{N}
\right) \hat{\mathrm{Var}}\left( t_{y, i} \right) =
\frac{N^2}{\left( n - 1 \right)n} \left( 1 - \frac{n}{N} \right) \sum_{i \in S}
\left( t_{y, i} - \hat{\bar{t_y}} \right)^2\), where \(\hat{\bar{t_y}} =
\frac{1}{n} \sum_{i \in S} t_{y, i}\).

Larger populations often mean that the totals for that population is also large,
for example when we are measuring API scores for schools, the total of all the
schools in a population will increase as the number of schools we sample
increases, as the score is always positive. This means that the variance of the
total estimates will usually increase if the clusters have very different
population sizes. It can therefore be useful to try to keep the populations in
the different clusters as similar as possible. But one should be careful not to put
too much weight on keeping the population sizes equal, as that may cause
the clusters to become too spread out physically which will make it expensive to
measure all individuals in it, which defeats the purpose of clustering
in the first place.

A cluster sample with the same number of individuals in it will
almost always have a larger variance than a SRS of the same size. This is
because individuals inside a cluster will usually be more similar than
individuals across clusters. This means that measuring say \(10\) people inside
one cluster will give less information than measuring \(10\) people randomly
chosen in the whole population, which will cause a larger uncertainty. Measuring
inside a cluster is often cheaper than measuring people sampled randomly from
the whole population, though. For example, if the clusters are geographical areas,
which they often are, and each individual sampled needs a researcher or
interviewer to physically travel to them, then one will save time and money
not having to travel to so many different places. This means that it is often possible to sample
many more individuals in cluster sampling than when doing samples from the whole
population, which will decrease the uncertainty and make cluster sampling more
competitive with other sampling methods.

\begin{example}
  %TODO: Fikse feil plot, kan løses ved å se på SE til slope coeffisient
  %istedenfor prediction intervall
  We again consider the API dataset and want to see if average class size for
  kindergarden through third grade
  impacts the API score for a school. Suppose that to collect the data, an
  interviewer has to travel to each school. Then it would take much time and
  become expensive if the interviewer had to travel to random schools all over
  California. It would be cheaper and take less time to
  collect data in just some school districts. We therefore choose to cluster
  on the school districts. There are \(757\) school districts in California and we take a SRS to
  get a sample of \(15\). This gives us a sample of \(183\) schools. Based on
  the data collected from these schools we can now make regression lines and
  confidence intervals for the regression coefficients.
  To make the regression line and to accurately estimate the variance we have to
  take the correlation between the individuals in the same cluster into account.
  If we do not take this into account we will get a different result and a too
  small variance estimate.

  \begin{table}
    \centering
    \begin{tabular}{c|ccc}
     & Slope & Std. Error & P-value \\ 
      \hline
      Clustering taken into account & 3.232 & 9.254 & 0.732 \\
      Clustering not taken into account & 3.232 & 0.593 & \(7.49 * 10^{-8}\) \\
    \end{tabular}
    \caption{Slope for regression lines along with uncertainty and P-value for
      the null-hypothesis that the slope is zero. The first row is for the case
      where the clustering is taken into account. In the second row the
      clustering was ignored, the samples were assumed to be independent.}
    \label{tab:exClust}
  \end{table}

  Observe from Table~\ref{tab:exClust} that the standard error of model where
  the clustering is ignored is significantly smaller than the standard error of
  the model where the clustering is accounted for. This is because we assume
  schools inside each district are more similar than schools in different
  districts. This causes a sample with many schools in the same districts to
  carry less information about the whole population than one where schools in
  all districts have a chance of being included in the sample. The different
  standard errors means that the hypothesis
  test, where the null-hypothesis is that the slope is zero, gives different
  results for the two models. Ignoring the clustering leads to the belief that larger average
  class size for kindergarden through third grade leads to a higher API score
  for the school. The sample does, however, suggests no such thing, as the P-value
  of the model where the clustering is taken into account is \(0.732\).

  
%  \begin{figure}
%    \centering
%    \includegraphics[scale = 0.5]{exClus}
%
%    \caption{Plot of regression line for a clustering sample. The red line is
%      the regression line from analyzing the sample as a clustered sample. The
%      blue line is the regression line from analyzing the clustered sample as an
%    SRS. The shaded areas are the regression lines 95\% prediction intervals.
%    The points are the data points from the whole population.}
%
%    \label{fig:exClus}
%  \end{figure}

  % Figure \ref{fig:exClus} shows two different regression lines. For the red
  % line, we accounted for clustering. We have taken
  % the sampling scheme into account, so the correlation between the individuals
  % in the same cluster is corrected for. For the blue line we ignored the
  % clustering and analyzed the sample as if it were a SRS. This ignores all
  % correlation between the individuals.

  % The shaded areas, which represent the 95\% prediction intervals, show that
  % taking the correlation between the sampled individuals into account gives a
  % larger variance than when we assume that the individuals are independently sampled.
  % This makes sense, as the clustered sample carries less information than the
  % SRS, as the individuals inside a cluster can usually be assumed to be more
  % similar than individuals in different clusters.

  % The 95\% prediction interval made from clustering includes \(\sim 97\%\) of the
  % population, while the 95\% prediction interval from interpreting the cluster
  % sample as a SRS includes just \(\sim 91\%\) of the population. Therefore, the 95\%
  % prediction interval for the regression line where we assume independence is
  % too small, which illustratess that it
  % is important to take the sampling method into account to get correct
  % uncertainty estimates.

  
\end{example}

\subsection{Complex surveys} \label{sec:complexSurveys}

In practice, we often combine clustering and stratification to what is called a
\textbf{complex survey}. We usually first create strata for the different subpopulations
we are interested in data from. Then we use clustering to make it
practically possible to perform the survey.

A real example, where survey statistics with a complex survey design is used, is the Demographic and Health
Survey (DHS) in Kenya. This survey collects, among other things, 
information regarding the birth rates and mortality rates in Kenya. This type of
survey is important because no one has any overview of vital statistics in
many developing countries.
Ideally, one would of course do a full census of the population, but that is not
possible in practice as it would be extremely expensive. Each household that is
surveyed, has to be visited by a trained interviewer. Visiting
households spread through the whole country would need both much travel time and
many interviewers who have to be trained. Instead one designs a
complex survey combining the sampling methods introduced earlier in Section
\ref{sec:RegFinPop}. This is cheaper, and if the survey is well designed
one will get good data from as well.

In the DHS in Kenya the researchers wanted data on both urban and rural
populations in each county, so they first started with splitting the country into
strata. Two strata for each county, one for the urban population and one for the
rural population, except for two counties which only have an urban population.
Each of these strata were then further divided into smaller geographic units,
these are the clusters. The researchers then do a SRS of the clusters inside
each stratum to choose which ones to visit. \cite{DHS}

Inside each cluster the researchers then sample 25 households to actually
visit. The fact that the researchers make a new sample inside the cluster is, as
mentioned in Section \ref{sec:clustering}
called two-stage clustering. It has not been a topic of this thesis, but the
principles used are the same as the ones used in one-stage clustering. \cite{DHS}

Counterintuively, one can often get better results doing a smaller well designed
survey than sampling as many as possible. This is because, if we have a small
sample, we need few interviewers and can therefore make sure
they are all well trained and suited to the job. If we just focus on getting as
large samples as possible one often has to compromise on the quality of the
training the interviewer get. This can lead to poorly trained interviewers introducing errors
in the data. For example, if no one from the household they are supposed to visit are
home, they might visit their neighbours instead of coming back later. This compromises the results of
the survey, as some demographics might be more likely to be home than others.

\section{Variance estimation} \label{sec:VarEst}

A central part of doing statistics is the calculation of uncertainties. We
usually express the uncertainty by estimating variance. Without an idea of the
variance of an estimated quantity, the estimate is all but useless, as the variance
could be so large that almost any confidence interval will include all possible
values. It is therefore important to have a reasonable estimate of the
variance. The problem, however, is that in regression the values we are
interested in are non-linear expressions of observations. We therefore can not find a closed form
expression for the variance estimate using normal rules, so we instead have to
use other techniques to
find variance estimates. One of the more commonly used methods in survey
statistics, and one which will give a closed form expression for the estimate,
is linearization.

\begin{example}
  Let \(\widehat{W} = \frac{\hat{t}_y}{\hat{t}_x}\), where \(\hat{t}_y\) and \(\hat{t}_x\) are stochastic variables. We
  are interested in finding \(\mathrm{Var} \left( \widehat{W} \right) = \mathrm{Var}
  \left( \frac{\hat{t}_y}{\hat{t}_x} \right)\).
  Since this is a non-linear expression, there is no easy formula to calculate
  it. We do, however, know how to calculate the variance of a linear
  combination. We can therefore take advantage of the fact that we can
  approximate \(\widehat{W} = \frac{\hat{t}_y}{\hat{t}_x}\) by using a first degree Taylor approximation,
  which is linear.

  Let \(h(a, b) = \frac{a}{b}\). Then \(\widehat{W} = h(\hat{t}_y, \hat{t}_x)\).
  \(h'(a, b) = \begin{bmatrix} \frac{1}{b} & -\frac{a}{b^2} \end{bmatrix}.\)

  This means that
  \begin{align*}
    \widehat{W} &= h(\hat{t}_y, \hat{t}_x)
    \approx h(t_y, t_x) + h'(t_y, t_x) \left( \begin{bmatrix} \hat{t}_y \\ \hat{t}_x \end{bmatrix}
    - \begin{bmatrix} t_y \\ t_x \end{bmatrix} \right)
    = \frac{t_y}{t_x} + \begin{bmatrix} \frac{1}{t_x} &
    -\frac{t_y}{t_x^2} \end{bmatrix} \begin{bmatrix} \hat{t}_y - t_y \\
      \hat{t}_x - t_x \end{bmatrix} \\
    &= \frac{t_y}{t_x} + \frac{1}{t_x} \left( \hat{t}_y - t_y \right) - \frac{t_y}{t_x^2} \left( \hat{t}_x - t_x \right)
    \end{align*}

    Since \(t_y\) and \(t_x\) are deterministic values, they have zero variance
    and we can treat them as any other constants in the variance expression. We can therefore apply
    the standard rules for calculating variances of linear expressions and we
    get that
    \begin{align*}
      \mathrm{Var} \left( \widehat{W} \right) \approx \mathrm{Var} \left(
      \frac{t_y}{t_x} + \frac{1}{t_x} \left( \hat{t}_y - t_y \right) -
      \frac{t_y}{t_x^2} \left( \hat{t}_x - t_x \right) \right)
    = \frac{1}{t_x^2} \mathrm{Var} \left( \hat{t}_y \right) +
      \frac{t_y^2}{t_x^4} \mathrm{Var} \left( \hat{t}_x \right) -
      2 \frac{t_y}{t_x^3}\mathrm{Cov}(\hat{t}_x, \hat{t}_y),
      \end{align*}
      which gives the simple expression \begin{align*}
                    \widehat{\mathrm{Var}} \left( \widehat{W} \right) = \frac{1}{\hat{t}_x^2} \widehat{\mathrm{Var}} \left( \hat{t}_y \right) +
                    \frac{\hat{t}_y^2}{\hat{t}_x^4} \widehat{\mathrm{Var}} \left( \hat{t}_x \right) - 2 \frac{\hat{t}_y}{\hat{t}_x^3}\widehat{\mathrm{Cov}}(\hat{t}_x, \hat{t}_y)
                    \end{align*}

    Taylor approximation works best if the point we are approximating around is
    near the point we want the function value at. This means that in this case we should
    choose a fixed point near \((\hat{t}_x, \hat{t}_y)\). Since this is an approximation
    of \((t_x, t_y)\), that point is a natural one to approximate around. Another
    advantage of approximating around \((t_x, t_y)\) is the fact that to get a
    variance approximation, we can exchange \((t_x,
    t_y)\) with \((\hat{t}_x, \hat{t}_y)\) and get a numeric value.
    Since \(\hat{t}_y\) and \(\hat{t}_x\) are linear expressions we are able to
    find closed form expressions for their variance estimates and we can
    therefore calculate \(\widehat{\mathrm{Var}} \left( \widehat{W} \right)\).

    To illustrate this with a concrete example, suppose \(\hat{t}_x = 100\) with
    \(\widehat{\mathrm{Var}}\left( \hat{t}_x \right) = 5\) and \(\hat{t}_y = 230\) with
    \(\widehat{\mathrm{Var}}\left( \hat{t}_y \right) = 23\) and that they are independent
    such that \(\mathrm{Cov} \left( \hat{t}_x, \hat{t}_y \right) = 0\).

    Then we have \(\widehat{\mathrm{Var}} \left( \widehat{W} \right) =
    \frac{1}{100^2} 23 + \frac{230^2}{100^4} 5 = 0.0049.\)

\end{example}

More generally, let \(h(x)\), where \(x = (x_1, x_2, \dots, x_p)\), be a function such that \(h(x) = \theta\), where
\(\theta\) is the statistic we are interested in. Further, let \(\hat{x}\) be an
estimate of \(x\), such that \(h(\hat{x}) = \hat{\theta}\). If \(h\) is
linear, there is no problem finding a closed form for the variance estimate,
so we assume here that \(h\) is a non-linear function.

By Taylor's theorem, we know that \(h(\hat{x}) = h(x) + h'(x) (\hat{x} - x) + \int_x^{\hat{x}}  (\hat{x} -
t) h''(t) dt\), \cite{kalkulus}, where the last term is usually small compared to the first two
in statistics, \cite[Chapter 9]{sampReg}. This allows us to use the approximation \(h(\hat{x}) \approx h(x) +
h'(x) (\hat{x} - x)\). \(h(x) = \begin{bmatrix} \frac{\partial h}{\partial x_1}
  (x_1) & \frac{\partial h}{\partial x_2} (x_2) & \dots & \frac{\partial
    h}{\partial x_p} (x_p)\end{bmatrix}\), which means that \(h(\hat{x}) \approx
h(x) + h'(x) (\hat{x} -
x) = h(x) + \sum_{i = 1}^p \frac{\partial h}{\partial x_i} (x_i) \left( \hat{x}_i - x_i
\right)\), which is a linear expression in \(\hat{x}\). Using normal rules for
variance estimation of linear expression we have
\begin{align*}
  \mathrm{Var} \left( \hat{\theta} \right)
  &= \mathrm{Var} \left( h(\hat{x}) \right)
  \approx \mathrm{Var} \left( h(x) + \sum_{i = 1}^p \frac{\partial h}{\partial x_i} (x_i) \left( \hat{x}_i - x_i
    \right) \right) \\
    &= \sum_{i = 1}^p \left( \frac{\partial h}{\partial x_i} (x_i)  \right)^2 \mathrm{Var} \left( \hat{x}_i \right) + \sum_{i \neq j}  \frac{\partial h}{\partial x_i} (x_i)  \frac{\partial h}{\partial x_j} (x_j)  \mathrm{Cov} \left( \hat{x}_i, \hat{x}_j \right)
\end{align*}
We do not, however, know \(x\), as that are the true values we are trying to
estimate. But by replacing \(x\) by \(\hat{x}\), and by replacing the variances
of \(\hat{x}\) by estimated variances in the expression we get an
estimate of the variance
\begin{align*}
  \widehat{\mathrm{Var}} \left( \hat{\theta} \right)
  = \sum_{i = 1}^p \left( \frac{\partial h}{\partial x_i} (\hat{x}_i)  \right)^2 \widehat{\mathrm{Var}} \left( \hat{x}_i \right) + \sum_{i \neq j}  \frac{\partial h}{\partial x_i} (\hat{x}_i)  \frac{\partial h}{\partial x_j} (\hat{x}_j)  \widehat{\mathrm{Cov}} \left( \hat{x}_i, \hat{x}_j \right)
\end{align*}

In the regression case, the differentiation and the estimates of covariance
are complicated, so we will not show the derivation here. However, it can be shown that
\begin{align*}
 \mathrm{Var}(\hat{B_1})
 &= \mathrm{Var} \left( h(\hat{t_{xy}}, \hat{t_x},
 \hat{t}_y, \hat{t}_{x^2}, \hat{N})) \right) \\
 &\approx \frac{\widehat{\mathrm{Var}}\left( \sum_{i \in S} w_i q_i \right)}
   {\left( \sum_{i \in S} w_i x_i^2 - \frac{\left( \sum_{i \in S} w_i x_i \right)^2}{\sum_{i \in S} w_i} \right)^2}
\end{align*}
where \(q_i = (y_i - \hat{B}_0 - \hat{B}_1 x_i)(x_i - \hat{\bar{x}})\), and
\(\bar{x} = \frac{\hat{t}_x}{\hat{N}}\), see \cite[Chapter 11]{sampReg} for more
details.


%Resampling hvis vi har tid

\section{Discussion} \label{sec:discussion}

% When analysing survey data it is important to take the survey design into
% account. In design based statistics have other things to take into account than
% we are used to in classical model based statistics. One big difference is the
% fact that since we assume a finite population it is possible to get no
% uncertainty for the regression line. If we sample the whole population we are
% 100\% certain of which linear line minimizes the RSS. This causes us to have a
% extra factor in the variance estimations called the finite population
% coefficient. This factor reduces the variance of our estimates as we sample
% larger parts of our population. For large populations this factor can often be
% discounted, but in smaller population or when we sample a large fraction of the
% population it helps reduce our uncertainty in our results.

% Another important thing to take into account is the fact that we don't always
% sample individuals independently. Sometimes we might sample some parts of the
% population with larger probabilities than others, and sometimes parts of our
% sample will be very correlated. This is important to take into account when
% making variance estimates. This happens mostly in clustering, where we sample
% only from some subsets of the population. This often causes more uncertainty
% than similarly sized independent samples, since each sample unit contains less
% information, as sampling units in the same clusters are usually similar.


%Tre hovedpunkter:
%* Bias, hva går feil, når, hvordan?

When using sampling designs where the sampling probabilities differ, we risk the
chance of having a bias in our results if we do not take the sampling design into account. This
is illustrated in Example \ref{ex:anthuneq} where the regression line where we ignore the
unequal sampling probabilities is too flat, and in Example \ref{ex:uneqProbs},
where we get different estimates of the totals. A bias in the regression line
can quickly make one reach conclusions, about the relationshiop between the
predictor and the response, not actually supported by the data.

%* Variansestimering er vankelig. Lett å lage estimater med vekter som er unbiased, men varians
%blir vanskelig. 

Since we can use weights, calculating estimates of most quantities is
straight forward, including estimates of regression coefficients. The non-linear nature
of the regression coefficients, however, makes it difficult to estimate the
variances. There are several methods to get variance estimates, including using
resampling, but the one used in this thesis is linearization. Using
linearization allows us to get reasonable variance estimates for the regression
coefficients. As seen in Section \ref{sec:VarEst} linearization has given us a
closed form expression for the variance estimates of the slope parameter, and a
similar one for the intercept is also possible to derive.

%
%* Varians
%  Hva skjer med varians når du har clustering?
%  Hvis du ikke tar høyde for det får du for lav varians og potensielt bias.
%  Hovedfeil: lav varians. Tror ting er uavhengige.
%  varianskorreksjon

Ignoring sampling design can cause major problems with variance estimation. In
many sampling designs clustering is used, which means sampling units in the sample are correlated. This means
that the sample carries less information than if the units were sampled independently.
Assuming that the sampling units in the sample are independent when they are not
will cause underestimation of the variance. Believing the variance is smaller
than it actually is can lead to a stronger belief in the results than warranted.
In regression one is often interested in whether the regression line is flat or
not, as a flat regression line means no relationship between predictor and response. Having a too small variance estimate might cause us to reject the null
hypothesis that the slope is zero, while we with a more correct variance would
not reject that null hypothesis. Ignoring the fact that one samples from a
finite population can also cause the opposite problem, believing that one has
more uncertainty than there actually is. This happens when one forgets to take
the fpc into account, which reduces the variance as one samples a larger
proportion of the population. This would reduce the power of the hypothesis test
as it would become harder to reject the null hypothesis, even when it is false.

%
%* Hvis iid, får ikke feil svar av survey statistics
%  Fordel med model based, trenger mindre data. Låner styrke med like parametre.
%  Blir vanskelig med survey statistics, kan ikke låne styrke fra naboer.
%  I model based er de korrelerte.
%  Har fylker, vil ha kommuner.
%
%  Hvis du har nok data bruk survey statistics, hvis man ikke har nok, må man
%  kanskje bruke model based.
%
%  Hvis modellen er helt riktig får man ingen bias av å ikke ta høyde for vekter
%  i model based. Men modeller er aldri helt riktig. 
%
%  Model based kan være nødvendig når man ikke har nok data. 
%
%  skal selvfølgelig bruke survey statistics, men må iblandt være pragmatisk.
%  Hvis respons er uavhengig av clusters, kan vi se bort fra dem. Tilsvarende med stratifisering.
%  Riktig svar: Samle mer data. Pragmatisk: Modle based.

It is important to note that using survey statistics will never give wrong
results. As illustrated in the bottom left panel of Figure \ref{fig:ex1}, the
regression line from classical linear regression is the exact same as the one
from survey statistics where we assume each observation is independently sampled.
This is also true in general. A disadvantage of survey statistics, however, is
that one in general needs more data to achieve good results. This is because in
model based statistics, the assumptions one make gives additional structure to
the data. This is structure which we would have to ``estimate'' using survey
statistics, causing us to ``use up'' some of the statistical power we want to
use for estimating regression coefficients and variance. If we have too little
data to get robust conclusions we would need to collect more data if we want
to keep using survey statistics. Another option, however, if it looks like the
assumptions regarding the distribution is  fulfilled, is to switch to
a model based regression. In addition, one would have to make sure that
clustering and stratification will not cause problems by violating independence.
If observations inside clusters are no more similar than observations in
different clusters one can disregard the clustering and still get good results.
One can correct for different sampling probabilities by incorporating weights in
the model based regression.
The correct choice is always to use survey statistics, but if you don't have
enough data and the model seems to fit it can sometimes be an acceptable
alternative to use model based regression instead.


%Ta opp viktigste poenger.
%Må gjøre varianskorreksjon.
%
%har ikke trukket tilfeldig.
%
%stratifisering
%
%clustering
%  Kan ikke gjøre dette i model based
%
%blir vanskelig å diskutere varians
%
%Veldig nyttig
%
%Ingen formler.
%
%Hvis du har gjort clustering og stratifisering er det direkte feil med model
%based. Se eksempel 1. Survey statistics er alltid riktig, model based noen ganger.
%
%Disse type metoder brukes ofte for å estimare dødelighet, vaksinasjonsrate og
%lignende i utviklingsland. Mest i clustering, litt her i diskusjonen.
%https://ntnuopen.ntnu.no/ntnu-xmlui/handle/11250/2624622 er master oppgave jeg
%kan se på og referere til. Spesielt interessant er 1.3.
%https://dhsprogram.com/pubs/pdf/fr308/fr308.pdf er rapport om Kenya som
%masteroppgaven over bruker. Kan referere til den.
%
%TODO: Du trenger en referanseliste til slutt og noen refereranser underveis. Spesielt i forbindelse med least squares estimatorene i Section 2, og noen av uttrykkene i Section 3. Det ville også vært fint å hatt en eller to referanser i Section 1.
%Natbib: citations, navn og årstall (authoryear)
%
%Cit seksjon 2, bok fra ST1101

\medskip

\bibliography{bibfile}

\end{document}