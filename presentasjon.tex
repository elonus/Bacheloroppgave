\documentclass{beamer}
\usepackage[utf8]{inputenc}

\title{Linear Regression for Survey Data}
\author{Andreas Matre \\
Supervisor: Geir-Arne Fuglstad}
\date{June 2020}

\begin{document}

\frame{\titlepage}

% Motivation

\begin{frame}
  \frametitle{Motivation example}

  % Plot med observasjoner og regresjonslinjer fra anthuneq datasett.

\end{frame}

\begin{frame}
  \frametitle{Problems with classical regression}

  \begin{itemize}
  \addtolength{\itemsep}{0.5\baselineskip}
  \item Assumes an infinite population % Kan føre til for stor varians
  \item Assumes that a stochastic distribution can describe the residuals % \
  \item Does not take sampling probabilities into account                 %  Kan
                                %  føre til bias
  \item Assumes a deterministic part of the relationship between predictor and
    respons which we try to estimate % To do that we minimize the RSS of the
                                % sample, impossible to find exactly
  \end{itemize}
  
\end{frame}

\begin{frame}
  \frametitle{Survey statistics}

  \begin{itemize}
  \addtolength{\itemsep}{0.5\baselineskip}
  \item Acknowledges a finite population
  \item Does not assume anything of the distribution of the residuals
  \item Takes the sampling design into account, for example unequal sampling
      probabilities
  \item Gets it's ``randomness'' from the probability of each unit being
      included in the sample
  \item There is one regression line which minimizes the RSS for the whole
    population % Can find it if we sample the whole population
  \end{itemize}
  
\end{frame}

% Method

\begin{frame}
  \frametitle{Regression}

  Notation: Assume population \((x_1, y_1), (x_2, y_2), \dots , (x_N, y_N)\),
  \(x_i, i = 1, \dots, N\) predictors, \(y_i, i = 1, \dots, N\) response. Sample
  \(S\) is set of indices of elements chosen.

  % Formler for regresjon med totaler
  \begin{itemize}
  \addtolength{\itemsep}{0.5\baselineskip}
  \item Can use same formulas as for normal regression, but rewritten:
  \item \(B_0 = \frac{1}{N} \left( t_y - \frac{t_{xy} t_x - \frac{1}{N} t_y t_x^2}
   {t_{x^2} - \frac{1}{N} t_x^2} \right) \)
  \item \(B_1 = \frac{t_{xy} - \frac{1}{N} t_y t_x}
   {t_{x^2} - \frac{1}{N} t_x^2} \)
  \item \(t_x = \sum_{i = 1}^N x_i\), \(t_y = \sum_{i = 1}^N y_i, t_{x^2} =
\sum_{i = 1}^N x_i^2\) and \(t_{xy} =
\sum_{i = 1}^N x_i y_i\)
  \item To estimate the coefficients we need to estimate the totals % We only do
                                % t_y, others are similar
  \end{itemize}
  % For å estimere trenger vi totaler
\end{frame}

\begin{frame}
  \frametitle{SRS}

  \begin{itemize}
  \addtolength{\itemsep}{0.5\baselineskip}
  \item Simple Random Sample
  \item All subsets of given size have same probability of being chosen
  \item \(\hat{t}_y = \frac{N}{n}\sum_{i \in S}y_i\) % Unbiased
  \item \(\mathrm{Var} \left( \hat{t}_y \right) = \frac{N^2}{n \left( N - 1 \right)} \left( 1 - \frac{n}{N} \right) \sum_{i = 1}^N (y_i - \bar{y})^2 \)
  \end{itemize}
  % Hva er SRS? Alle delmengder av valgt størrelse har lik sannsynlighet for å
  % bli valgt
  % Estimatorer for SRS, forventningsrette
\end{frame}

\begin{frame}
  \frametitle{Stratification}

  \begin{itemize}
  \addtolength{\itemsep}{0.5\baselineskip}
  \item Want separate estimates for subpopulations % Can risk not having enough
                                % sampled units in the different subpopulations
                                % to get good estimates
  \item Sample from each subpopulation independently
  \item \(\hat{t}_{y} = \sum_{h = 1}^H\hat{t}_{h, y}\) % Different sampling
                                % methods inside strata gives different estimates
  \item SRS: \(\hat{t}_{h, y} = \frac{N_h}{n_h} \sum_{i \in S_h}y_i\)
  \item Homogenous strata reduces variance
  \end{itemize}

  % Ønsker estimater for forskjellige deler av populasjonen
  % Kan risikere å ikke få med alle deler man vil
  % Kan redusere varians hvis strata er like innad

  % Estimatorer
  % Vekter
  
\end{frame}

\begin{frame}
  \frametitle{Weights}

  \begin{itemize}
  \addtolength{\itemsep}{0.5\baselineskip}
  \item Makes calculations easier
  \item Number of elements a sampled unit represents
  \item Inverse of sampling probability
  \item \(\hat{t}_{y} = \sum_{i = S}w_iy_i\)
  \end{itemize}
  
\end{frame}

\begin{frame}
  \frametitle{Clustering}

  \begin{itemize}
  \addtolength{\itemsep}{0.5\baselineskip}
  \item Can be expensive to sample people randomly
  \item Split population into subpopulations, sample only inside some of these
  \item One-stage, two-stage
  \item \(\hat{t}_y = \frac{N}{n} \sum_{i \in S} \hat{t}_{y, i} = \sum_{i \in
      S_1 \cup S_2 \cup \dots \cup S_n} w_i y_i\)
  \item \(\widehat{\mathrm{Var}}\left(t_{y}\right) = \frac{N^2}{n} \left( 1 - \frac{n}{N}
\right) \widehat{\mathrm{Var}}\left( t_{y, i} \right)\)
  \item Variance increases
  \item Can affort to sample more units
  \end{itemize}
  % Dyrt å reise rundt, vil spare penger ved å ikke reise helt tilfeldig rundt.
  % Sampler clusters, og så sampler igjen innad
  % Øker varians, siden hver sample har mindre informasjon siden enheter er avhengige
  % Kan sample fler for samme pris
  % Vil ha clusters som er forskjellige innad
  % 
  
\end{frame}

\begin{frame}
  \frametitle{Linearization}

  \begin{itemize}
  \addtolength{\itemsep}{0.5\baselineskip}

\item Non-linear expression -> Hard to estimate variance % Very important to
                                % estimate uncertainties, otherwise we know nothing
  \item Can use linearization
  \end{itemize}

  % Vanskelig å estimare varians, siden koeffisientene er ikke-lineære uttrykk
  % Bruker Taylor approksimasjon rundt den sanne verdien
  % Bytter ut sann verdi med estimat
  % Lettere å regne ut estimatet, siden istedenfor Var(B_1) har vi Var(t_q) som
  % er lineær som funksjon av q
\end{frame}

% Resultater

\end{document}